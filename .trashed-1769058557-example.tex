%%%%%%%%%%%%%%%%%%%%%%%%%%%%%%%%%%%%%%%%%%%%%%%%%%%%%%%%
% LaTeX Notebook Template for Math & Physics
% A template for students (like those at ESFM-IPN)
% who need to take structured, rigorous notes.
%%%%%%%%%%%%%%%%%%%%%%%%%%%%%%%%%%%%%%%%%%%%%%%%%%%%%%%%

% Set the document class. '12pt' is readable, 'letterpaper' is standard in the Americas.
\documentclass[12pt, letterpaper]{article}

%=========== PREAMBLE: PACKAGES AND SETUP ===========

% 1. FONT & ENCODING
\usepackage[utf8]{inputenc} % Handle UTF-8 input characters
\usepackage[T1]{fontenc}      % Use modern 8-bit font encoding
\usepackage{lmodern}          % Use the Latin Modern font (a nice default)

% 2. PAGE LAYOUT
\usepackage[margin=1in]{geometry} % Set 1-inch margins on all sides

% 3. ESSENTIAL MATH & PHYSICS PACKAGES
\usepackage{amsmath}    % American Mathematical Society's (AMS) math enhancements
\usepackage{amssymb}    % AMS extra math symbols (like \mathbb{R})
\usepackage{amsfonts}   % AMS math fonts
\usepackage{amsthm}     % AMS theorem environments (for proofs, definitions)
\usepackage{physics}    % Excellent package for physics notation (vectors, derivatives, etc.)

% 4. UTILITY PACKAGES
\usepackage{graphicx}         % For including images (\includegraphics)
\usepackage{xcolor}           % For defining custom colors
\usepackage[colorlinks=true, allcolors=blue]{hyperref} % Clickable links (ToC, refs)
\usepackage{booktabs}         % For professional-looking tables (\toprule, \midrule)
\usepackage{parskip}          % Use space between paragraphs instead of indentation

% 5. METADATA
\title{}
\author{Morales Guerra Ervin Omar} % <-- EDIT THIS
\date{\today}           % Automatically uses today's date

%=========== CUSTOM THEOREM ENVIRONMENTS ===========
% This section sets up the styles for definitions, theorems, etc.

\theoremstyle{definition} % Upright text, bold heading
\newtheorem{definition}{Definition}[section] % Numbered as Definition 1.1, 1.2, etc.
\newtheorem{example}[definition]{Example}    % Shares numbering with definitions
\newtheorem{problem}[definition]{Problem}    % Shares numbering with definitions

\theoremstyle{plain} % Italicized text, bold heading
\newtheorem{theorem}[definition]{Theorem}      % Shares numbering
\newtheorem{lemma}[definition]{Lemma}          % Shares numbering
\newtheorem{corollary}[definition]{Corollary}    % Shares numbering
\newtheorem{proposition}[definition]{Proposition} % Shares numbering

\theoremstyle{remark} % Upright text, italic heading
\newtheorem{remark}[definition]{Remark} % For side notes or observations

% The 'proof' environment is already defined by amsthm
% It automatically adds a Q.E.D. symbol (□) at the end.

%=========== CUSTOM MATH SHORTCUTS (MACROS) ===========
% Define your own commands here to save time

\newcommand{\R}{\mathbb{R}} % Shortcut for Real Numbers
\newcommand{\C}{\mathbb{C}} % Shortcut for Complex Numbers
\newcommand{\Z}{\mathbb{Z}} % Shortcut for Integers
\newcommand{\N}{\mathbb{N}} % Shortcut for Natural Numbers
\newcommand{\Q}{\mathbb{Q}} % Shortcut for Rational Numbers

% Physics shortcuts
\newcommand{\Lag}{\mathcal{L}} % Lagrangian
\newcommand{\Ham}{\mathcal{H}} % Hamiltonian


%=========== DOCUMENT BODY ===========================
\begin{document}
		
	\maketitle % Display the title, author, and date
	
	\tableofcontents % Generate a Table of Contents
	\newpage
	
	%=========== START OF NOTES ==========================
	
	\section{First Topic (e.g., Classical Mechanics)}
	
	\subsection{Newton's Laws}
	We begin with a formal definition of force.
	
	\begin{definition}[Inertial Frame]
		A frame of reference in which Newton's first law holds is called an **inertial frame**. That is, an object with no net force acting on it will move with constant velocity.
	\end{definition}
	
	\begin{theorem}[Newton's Second Law]
		The net force $\vb{F}$ acting on a particle of mass $m$ is proportional to its acceleration $\vb{a}$.
		\begin{equation}
			\vb{F} = m\vb{a}
			\label{eq:newtons_second}
		\end{equation}
		Using the `physics` package, we can also write this using derivatives:
		\begin{equation}
			\vb{F} = \dv{\vb{p}}{t} = \dv{}{t}(m\vb{v})
		\end{equation}
	\end{theorem}
	
	\begin{proof}
		The proof is based on experimental observation. We observe that for a given body, the ratio of force magnitude to acceleration magnitude is constant. We define this constant as the inertial mass, $m$.
	\end{proof}
	
	\begin{remark}
		Equation \eqref{eq:newtons_second} is a vector equation, meaning it is equivalent to three scalar equations: $F_x = ma_x$, $F_y = ma_y$, and $F_z = ma_z$.
	\end{remark}
	
	\section{Second Topic (e.g., Linear Algebra)}
	
	\subsection{Vector Spaces}
	Here is an example of a multi-line equation using the `align*` environment (the `*` means it won't be numbered).
	
	\begin{example}[Maxwell's Equations]
		A famous example of a coupled system of partial differential equations:
		\begin{align*}
			\div{\vb{E}} &= \frac{\rho}{\epsilon_0} \\
			\div{\vb{B}} &= 0 \\
			\curl{\vb{E}} &= -\pdv{\vb{B}}{t} \\
			\curl{\vb{B}} &= \mu_0 \left( \vb{J} + \epsilon_0 \pdv{\vb{E}}{t} \right)
		\end{align*}
		%%		Note the use of `\div{}`, `\curl{}`, and `\pdv{}{}` from the `physics` package.
	\end{example}
	
	\subsection{Eigenvalue Problems}
	We often seek solutions to the equation $A\vb{v} = \lambda\vb{v}$.
	
	\begin{proposition}
		If $A$ is an $n \times n$ Hermitian matrix, its eigenvalues are real.
	\end{proposition}
	
	\begin{proof}
		Let $A$ be a Hermitian matrix, so $A = A^\dagger$. Let $\lambda$ be an eigenvalue with corresponding eigenvector $\vb{v}$.
		$$ A\vb{v} = \lambda\vb{v} $$
		Taking the Hermitian conjugate of $\vb{v}$ and multiplying:
		$$ \vb{v}^\dagger A \vb{v} = \lambda \vb{v}^\dagger \vb{v} $$
		Now, take the Hermitian conjugate of the first equation:
		$$ (A\vb{v})^\dagger = (\lambda\vb{v})^\dagger \implies \vb{v}^\dagger A^\dagger = \lambda^* \vb{v}^\dagger $$
		Since $A = A^\dagger$, this becomes $\vb{v}^\dagger A = \lambda^* \vb{v}^\dagger$. Multiply on the right by $\vb{v}$:
		$$ \vb{v}^\dagger A \vb{v} = \lambda^* \vb{v}^\dagger \vb{v} $$
		Comparing the two results, we have:
		$$ \lambda \vb{v}^\dagger \vb{v} = \lambda^* \vb{v}^\dagger \vb{v} $$
		Since $\vb{v}$ is an eigenvector, $\vb{v}^\dagger \vb{v} = \norm{\vb{v}}^2 \neq 0$. Therefore, we can divide by it, leaving:
		$$ \lambda = \lambda^* $$
		This is only true if $\lambda$ is a real number.
	\end{proof}

	
\end{document}
